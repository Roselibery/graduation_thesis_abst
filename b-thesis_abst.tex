\documentclass[10pt]{jarticle}
\usepackage{float}
\usepackage{adrobo_abst}
\usepackage[dvipdfmx]{graphicx}
\usepackage{amssymb,amsmath}
\usepackage{bm}
\usepackage[superscript]{cite}
\usepackage{enumerate}
\usepackage{url}
%\usepackage[absolute]{textpos}

\renewcommand\citeform[1]{(#1)}

\begin{document}
    
    \makeatletter
    \doctype{2025年度卒業論文概要}
    \title{ラプラシアン分散を用いたコマの回転停止検出}{}
    \etitle{Image-Based Detection of Spinning Top Stopping State using Laplacian Variance} {}

    \author{22C1061\hspace{.5zw}澤田悠李}
    \eauthor{Yuri Sawada}
    
    \makeatother
    
    \abstract{This paper proposes a method for detecting the rotational and stopping states of a spinning object using an image-based blur metric derived from the Laplacian variance. The Laplacian operator emphasizes high-frequency components such as edges, which makes its variance a widely used indicator of image sharpness. In this study, video frames of a spinning top are analyzed, and the temporal variation of the blur value is evaluated. Experimental results show that the blur value strongly depends on rotational speed and object posture, and that a fixed theoretical threshold does not always achieve reliable stopping detection. Based on these observations, we discuss the limitations of purely theory-based threshold determination and the necessity of experimentally calibrated decision criteria.}
   
    \keywords{blur detection, Laplacian operator, rotation detection, computer vision}
    
    \maketitle
    
    \supervisor{指導教員:上田隆一 教授}
    
    \section{緒\hspace{2zw}言}%===========================
 
	近年,ベイブレードを用いた競技大会においては,勝敗判定や試合進行の公平性を確保することが重要な課題となっている.しかし,実際の大会では,人間の審判が目視によりコマの回転状態や停止の有無を判断しており,高速回転中や停止直前の挙動を正確に判別することは容易ではない.特に,複数のコマが同時に動作する状況や,回転速度が低下した状態では,判定のばらつきや見落としが生じる可能性がある.このような背景から,画像処理技術を用いた自動審判システムの実現が期待されている.
	
	ベイブレードに関連する研究として,赤外線カメラとプロジェクタ投影を用いたタンジブルインタフェースの開発が報告されている\cite{beysion2024}.この研究では,スタジアム上のコマ位置を検出し,視覚的な演出によって競技体験を向上させることを目的としており,コマの検出および追跡手法は本研究においても参考とした.しかし,同研究ではコマの回転状態や停止状態の判定までは行われておらず,審判機能として用いるには不十分である.
	
	一方,画像中のぼけ量を定量化する手法は,オートフォーカスや画質評価をはじめとして広く研究されており,特に Laplacian 演算子の分散(LaplacianVariance)は,画像中の高周波成分を簡便に評価できる指標として知られている\cite{bansal2016}.回転体を撮影した場合,回転速度や姿勢変化に応じて画像中のぼけ量が変化することから,この指標を用いることで回転状態の推定が可能であると考えられる.
	
	そこで本研究では,ベイブレードの自動審判システムの実現を最終目標として,コマの回転映像を対象に,Laplacian に基づく blur 指標を用いた回転状態および停止状態の検知手法について検討する.特に,理論的に導出した閾値と実験的に得られた閾値との差異に着目し,単一の閾値による判定の限界と,実環境における課題を明らかにすることを目的とする.
     
    \section{提案手法}%===========================
    各フレームをグレースケール画像に変換した後,以下の手順で blur 指標を算出する.
    \begin{enumerate}
	    \item 入力画像をグレースケール化
	    \item Laplacian 演算子を適用
	    \item 出力画像の分散を blur 値として算出
    \end{enumerate}

    Laplacian による blur 指標 $B$ は次式で定義される(1).

    \begin{equation}
	    B = \mathrm{Var} \left( \nabla^2 I \right)
    \end{equation}

    ここで $I$ は入力画像,$\nabla^2$ は Laplacian 演算子を表す.一般に,画像が鮮鋭であるほど $B$ は大きくなり,ぼけが強い場合には小さな値を取る.本研究では,連続フレームにおける $B$ の時間変化に注目し,回転状態と停止状態の判別を行う.

    \section{実験}%===========================
	本研究では,提案手法の有効性とその限界を検証するため,コマの回転映像を用いた実験を行った.実験1では,Laplacian に基づく blur 指標の時間変化を分析し,回転状態と停止状態における基本的な振る舞いを確認する.
次に実験2では,blur 値に時間窓平均を適用することで,判定の安定化が可能であるかを検討する.
最後に実験3では,blur に加えて位置変動および半径変動といった複数の特徴量を統合し,実環境における停止状態検知の性能を評価する.


    	\subsection{撮影方法}
		図のようにコマを回すスタジアムの上にスマートフォンを設置し,超広角60fpsでコマを回し始めてから回転が完全に止まるまでの撮影を行った.
		\begin{figure}[h]                                                                                                                                                   \begin{center}
                                \includegraphics[width=0.5\linewidth]{fig/experiment.jpg}
                                \caption{Recording equipment}
                        \end{center}
                \end{figure}
	\subsection{実験1:Blur 特徴量の時系列分析}
    		回転から停止までの動画を撮影し,各フレームの blur 値を算出して時系列的に分析した.回転停止前後における blur 値の時系列変化を図2に示す.		
		回転中,blur 値は概ね 350〜450 程度の範囲で推移し,比較的安定した値を示していることが分かる.一方,停止直前の減速過程において,blur 値が一時的に低下した後,停止時刻付近で急激に上昇する挙動が確認された.停止後は,blur 値の平均が回転中よりも高い値を示すものの,フレームごとのばらつきが大きく,一部のフレームでは回転中と同程度の値を取る場合も観測された.

		\begin{figure}[h]
                        \begin{center}
                                \includegraphics[width=1\linewidth]{fig/Figure_02.png}
				\caption{Blur around rotation stop}
                        \end{center}
                \end{figure}
		
	\subsection{実験2:時間窓を用いた判定}
		blur 値に時間窓を導入し,平均値に基づく判定を行った.これにより瞬間的な誤判定は減少したが,低速回転時の誤判定は完全には解消されなかった.
    	
	\subsection{実験3:複数特徴量の統合}
		blur 値に加えて,コマの位置変動および半径変動を統合した判定を行った.

		傾いた停止状態や低速回転時においても停止判定が可能となった.

		\begin{figure}[H]
			\centering
			\begin{minipage}[t]{0.48\linewidth}
				\centering
				\includegraphics[width=\linewidth]{fig/experiment3_spin.png}
				\caption{Spining}
			\end{minipage}
			\hfill
			\begin{minipage}[t]{0.5\linewidth}
				\centering
				\includegraphics[width=\linewidth]{fig/experiment3_stop.png}
				\caption{Stoped}
			\end{minipage}
		\end{figure}

    \section{結\hspace{2zw}言}%===========================
    本研究では,Laplacian に基づく blur 指標を用いて,回転体の停止状態検知を試みた.実験の結果,理論に基づく閾値設定のみでは,実際の回転挙動を十分に反映できない場合があることを示した.

    また,blur 値に時間窓平均を適用することで判定の安定化は得られたものの,特徴量の分離性は改善されず,停止判定には至らなかった.一方で,位置変動や半径変動といった複数の特徴量を統合することで,実環境においても比較的安定した停止判定が可能となった.

以上より,単一の画像特徴量に基づく判定には限界があり,自動審判システムの実現には複数の視覚情報を統合した判定が有効であることを示した.
%@@@ここレイアウトくずれてました@@@

    \vspace{5truemm}
    {\footnotesize
        \begin{thebibliography}{99}
		
		\bibitem{bansal2016}
			R. Bansal, G. Raj, and T. Choudhury,
			``Blur Image Detection using Laplacian Operator and Open-CV,''
			Proc.SMART 2016,pp.~1--5, 2016. 

		\bibitem{beysion2024}
			辻 大朗, 氷見 和, 森 春介, 山名 巧真, 高橋 李奈, 越野 亮,
			「Beysion: 赤外線カメラとプロジェクタ投影によるタンジブルインタフェースの開発」,
			\textit{第29回日本バーチャルリアリティ学会大会論文集},2B1-02, 2024.


        \end{thebibliography}
    }
    \normalsize
    
\end{document}
